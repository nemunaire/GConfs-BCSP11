\chapter{Présentation de la programmation orientée objet}

\section{Le concept}

L'idée directrice de la programmation orientée objet est de regrouper les données au sein d'objets tels qu'on les voit~: une voiture, un poison, un animal, \ldots

Ces données sont les caractéristiques de ces objets~; par exemple, la voirutre a une couleur, un nombre de portes, une vitesse, \ldots~; un animal a une couleur, une taille, un poids.

Chacune de ces caractéristiques peut être modifiée par une ou plusieurs fonctions. Par exemple, la voiture a une fonction accélérer et ralentir qui modifie sa vitesse. Un animal peut quant-à lui avoir une fonction manger qui augmente sa taille et son poids.

Chaque objet possède donc des caractéristiques et des fonctions. Comme tous les objets que l'on rencontre tous les jours~!

\helpbox{Info}{on appelle les caractéristiques des objets sont quant-à elles appelées \og méthodes\fg.}

\section{Accessibilité}

Pour chaque élément (caractéristique, fonction ou classe), vous devez définir
son niveau de visibilité. Les deux niveaux les plus courants sont \verb+public+
et \verb+private+. Lorsqu'un élément est définit comme \verb+public+, il est
accessible et modifiable (sauf les méthodes) depuis n'importe quelle autre
classe. Lorsque vous définissez un élément comme \verb+private+, il accessible
et modifiable uniquement depuis la classe qui les définie.

\section{Les différent types de variables}

Pour rappeler, une variable permet de stocker une donnée d'un type définie
dans la mémoire vive de l'ordinateur, un entier par exemple. Ces données
peuvent être simples~: un entier, un caractère, un peu plus évoluées~: un
chaîne de caractères, une position dans le plan ou encore plus complexe~: un
moteur graphique, physique \ldots

Ces trois types de variables servent à stocker et à accéder à ces valeurs.

\subsection{Les attributs}

Ce sont les \og véritables variables \fg. La quasi-totalité du temps elles sont
déclarées privé car on préfère les modifier depuis des méthodes ou des
accesseurs. Elles se déclarent de cette façon, juste après le début d'une
classe~:

\begin{lstlisting}
private int myVar;
\end{lstlisting}

\subsection{Les propriétés}

Ce sont des attributs \og améliorés \fg. Il est possible de déclarer des
propriétés publiques et de limiter la lecture ou l'écriture depuis les autres
classes. Elles se déclarent de cette façon~:

\begin{lstlisting}
public int MyProperty
{
  get;
  private set;
}
\end{lstlisting}

Si on enlève le \verb+private+ devant le \verb+set+, il est alors possible de
modifier la variable depuis une autre classe.

\subsection{Les accesseurs}

À cheval entre les méthodes et les propriétés, les accesseurs se déclarent de
la même manière que les propriétés mais ont un fonctionnement proche des
méthodes. Par exemple, pour accéder à notre attribut \verb+myVar+, déclarée
plus haut, depuis une autre classe, on écrira~:

\begin{lstlisting}
public int MyAcc
{
  get
  {
    return this.myVar;
    }
}
\end{lstlisting}

Vous pouvez également définir la partie \verb+set+ lorsque vous souhaitez
pouvoir définir l'attribut en question. Ce qui donne l'accesseur suivant~:

\begin{lstlisting}
public int myAcc
{
  get
  {
    return this.myVar;
    }
    set
    {
      this.myVar = value;
      }
}
\end{lstlisting}

Le mot clé \verb+value+ correspond à la valeur passé à l'accesseur lorsqu'il
est appelé~:

\begin{lstlisting}
MyClasse instance = new myClass();
instance.MyVar = 42;
\end{lstlisting}

Dans cet exemple, \verb+value+ vaudra donc 42.

Il est tout à fait possible (et c’est même encouragé) de tester la valeur passé
à l'accesseur. Vous pouvez en effet faire tous les tests que vous voulez mais
aussi modifier d'autres attributs ou propriétés de manière à ce qu'elles
restent toutes cohérentes.

Par exemple, remettre à 0 une variable lorsque vous en modifiez une autre~:
\begin{lstlisting}
public int myAcc
{
  get
  {
    return this.myVar;
    }
    set
    {
      if (value > 0)
      {
        this.myVar = value;
        this.incr = 0;
        }
        }
}
\end{lstlisting}

Ici, l'attribut \verb+myVar+ n'est modifié que si la valeur passé est
supérieure à 0 et dans ce cas, l'attribut \verb+incr+ est définit à 0.

\section{Instanciation de classes}

Commençons par rappeler la déclaration d'une variable de base~:

\begin{lstlisting}
int a = 42;
\end{lstlisting}

On indique ici au compilateur que la variable \verb+a+ est de type entier
(\verb+int+) et vaut 42 à son initialisation. En effet chaque variable doit
avoir un type défini. Les structure et les classes sont des types de données.

Prenons comme exemple la classe Point. Pour créer un nouveau point, on fait
ainsi~:

\begin{lstlisting}
Point b = new Point(2, 4);
\end{lstlisting}

Ici, le premier \verb+Point+ indique le type de la variable \verb+b+. Le deuxième initialise la variable en appelant le constructeur de \texttt{Point} avec les deux arguments qu'il attend.