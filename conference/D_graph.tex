\section{Moteur graphique}

\subsection{Rôle}
\begin{frame}{Moteur graphique}
  \begin{block}{Rôle}
    \begin{itemize}
      \item La sortie principale
      \item Fourni au joueur un apperçu sur l’état du jeu
      \item Doit être fiable et précis
      \item Apporte une partie du fun
    \end{itemize}
  \end{block}
  % Pour un jeu, la partie la plus importante c’est le moteur graphique, c’est
  % celle qui donne le premier apperçu de votre jeu.
  % Pour le joueur c’est le principal moyen, si ce n’est le seul, de savoir où
  % il est en est et quel est le résultat de ses actions, ce moteur doit être
  % fiable et réactif.
  % Une partie du fun est transmis par lui, des effets kikoo et des machins
  % qui explosent de partout c’est toujours marrant.
\end{frame}

\subsection{Fonctionnement général}
\begin{frame}
  \begin{block}{While true}
    \begin{itemize}
      \item Les données du jeu doivent être mises à jour en permanence
      \item Une solution : appeler une fonction de mise à jour en boucle
      plusieurs dizaine de fois par seconde.
      \item De même pour les graphismes~: image à l'écran raffraichie
      plusieurs dizaine de fois par seconde.
      \item Séparation données mémoire du jeu / affichage.
    \end{itemize}
  \end{block}
\end{frame}
% un while true géant
% séparation update,données en mémoire/draw, trucs affichés
% sprite, tileset, double buffering, 2D basics
% 3D basics

\subsection{Frameworks et bibliothèques}
\begin{frame}
  \begin{block}{Bibliothèques et frameworks}
    \begin{itemize}
      \item XNA
      \item DirectX
      \item SDL
      \item OpenGL
      \item NSREGX
      \item ...
    \end{itemize}
  \end{block}
\end{frame}

\subsection{XNA}

\begin{frame}
  Fait le café.
  % organisation : update, draw
  % graphisme 2D : spritebatch + content pipeline
  % graphisme 3D : ?
  % interractions : clavier souris
  % son : beurk, au dernières nouvelles
  % réseau : noooooon pas le Windows Live Game
\end{frame}
